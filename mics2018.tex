% This is sigproc-sp.tex -FILE FOR V2.6SP OF ACM_PROC_ARTICLE-SP.CLS
% OCTOBER 2002
%
% It is an example file showing how to use the 'acm_proc_article-sp.cls' V2.6SP
% LaTeX2e document class file for Conference Proceedings submissions.
%
%----------------------------------------------------------------------------------------------------------------
% This .tex file (and associated .cls V2.6SP) *DOES NOT* produce:
%       1) The Permission Statement
%       2) The Conference (location) Info information
%       3) The Copyright Line with ACM data
%       4) Page numbering
%
%  However, both the CopyrightYear (default to 2002) and the ACM Copyright Data
% (default to X-XXXXX-XX-X/XX/XX) can still be over-ridden by whatever the author
% inserts into the source .tex file.
% e.g.
% \CopyrightYear{2003} will cause 2003 to appear in the copyright line.
% \crdata{0-12345-67-8/90/12} will cause 0-12345-67-8/90/12 to appear in the copyright line.
%
%
%---------------------------------------------------------------------------------------------------------------
% It is an example which *does* use the .bib file (from which the .bbl file
% is produced).
% REMEMBER HOWEVER: After having produced the .bbl file,
% and prior to final submission,
% you need to 'insert'  your .bbl file into your source .tex file so as to provide
% ONE 'self-contained' source file.
%
% Questions regarding SIGS should be sent to
% Adrienne Griscti ---> griscti@acm.org
%
% Questions/suggestions regarding the guidelines, .tex and .cls files, etc. to
% Gerald Murray ---> murray@acm.org
%
% For tracking purposes - this is V2.6SP - OCTOBER 2002


\documentclass[12pt]{article}

\setlength{\oddsidemargin}{0in}
\setlength{\evensidemargin}{0in}
\setlength{\topmargin}{0in}
\setlength{\headheight}{0in}
\setlength{\headsep}{0in}
\setlength{\textwidth}{6in}
\setlength{\textheight}{9in}
\setlength{\parindent}{0in}

\usepackage{graphicx} %For jpg figure inclusion
\usepackage{times} %For typeface
\usepackage{epsfig}
\usepackage{color} %For Comments
%\usepackage[all]{xy}
\usepackage{float}
%\usepackage{subfigure}
\usepackage{hyperref}
\usepackage{url}
\usepackage{parskip}

%% Elena's favorite green (thanks, Fernando!)
\definecolor{ForestGreen}{RGB}{34,139,34}
\definecolor{BlueViolet}{RGB}{138,43,226}
\definecolor{Coquelicot}{RGB}{255, 56, 0}
\definecolor{Teal}{RGB}{2,132,130}
%Uncomment this if you want to show work-in-progress comments
\newcommand{\comment}[1]{{\bf \tt  {#1}}}
% Uncomment this if you don't want to show comments
%\newcommand{\comment}[1]{}
\newcommand{\emcomment}[1]{\textcolor{ForestGreen}{\comment{Elena: {#1}}}}
\newcommand{\todo}[1]{\textcolor{blue}{\comment{To Do: {#1}}}}
\newcommand{\tscomment}[1]{\textcolor{Teal}{\comment{Tony: {#1}}}}
%%%%%%%%%%%%%%%%%%%%%%%%%%%%%%%%%%%%%%%%%%

\begin{document}
\pagestyle{plain}
%
% --- Author Metadata here ---
%\conferenceinfo{WOODSTOCK}{'97 El Paso, Texas USA}
%\setpagenumber{50}
%\CopyrightYear{2002} % Allows default copyright year (2002) to be
%over-ridden - IF NEED BE.
%\crdata{0-12345-67-8/90/01}  % Allows default copyright data
%(X-XXXXX-XX-X/XX/XX) to be over-ridden.
% --- End of Author Metadata ---

\title{TBA}
%\subtitle{[Extended Abstract \comment{DO WE NEED THIS?}]
%\titlenote{}}
%
% You need the command \numberofauthors to handle the "boxing"
% and alignment of the authors under the title, and to add
% a section for authors number 4 through n.
%
% Up to the first three authors are aligned under the title;
% use the \alignauthor commands below to handle those names
% and affiliations. Add names, affiliations, addresses for
% additional authors as the argument to \additionalauthors;
% these will be set for you without further effort on your
% part as the last section in the body of your article BEFORE
% References or any Appendices.

\author{
Charlot Shaw \\
Computer Science Discipline \\
University of Minnesota Morris\\
Morris, MN 56267\\
shawx538@morris.umn.edu
}
\maketitle
\thispagestyle{empty}
%\alcomment{Should these say @morris.umn.edu?}

\section*{\centering Abstract}


\newpage
\setcounter{page}{1}

\section{Introduction}
Clojure~\cite{Hickey:2008}, amongst other functional languages, have gained attention in recent years
in part for its management of state and concurrency.
Runing on the Java Virtual Machine (JVM), Clojure has a number of advantages for beginning students.
Its Lisp style syntax and elegant core library
 work together to let beginners learn easily, while
its ties to the JVM keep it relevant as students progress into later courses.

However, from an educational perspective, it has a significant flaw in the form of error messages.
As Clojure code runs in the JVM, its errors take the structure and terminology of Java error messages,
and so are confusing to new students. They can understand the source of the error,
but not how the system presents it to them.
For example, a user who accidentally called addition on a boolean needs
 to understand the Java object hierarchy, casting,
 and the classes involved to fully understand what is meant by
 ``java.lang.classCastException java.lang.Boolean cannot be cast to java.lang.Number'',
whereas the background required to understand ``In function + the first argument true must be a number, but is a boolean''
 is significantly lower.
 In order to overcome these problems,
 we have explored possibilities for integrating customized error messages with commonly used tools in the Clojure community.
%  we demonstrate a proof of concept integration with commonly used tools in the Clojure community
% to transform error messages into simplified forms before they are presented to the user.
%\emcomment{we have explored possibilities for integrating custom-made error messages with commonly used tools in the Clojure community}




\section{Error Messages in Clojure}
Clojure is hosted and interpretted in the JVM, as a Java program.
Clojure code can either be loaded by a running Clojure process, or
compiled Ahead Of Time (AOT compilation) into Java bytecode. In either case,
even simple Clojure code mid execution is a very complex Java program,
and any exceptions thrown are thrown as Java errors.
From the viewpoint of the JVM, the entirety of the Clojure process is
part of the users program, and so error messages include large amounts
of data about the underlying state of the Java classes that implement Clojure.
What information regarding the error that remains unobscured is still
phrased as a Java error, forcing a Clojure user into learning
the underlying Java code. This undue onus on the beginner,
unfamiliar with programming in general is unreasonable.
However, the large amounts of information in a Clojure error
can be part of processing down the error for begginers.

%Clojure is hosted and interpretted in the JVM, with Clojure code either being
%loaded by a Clojure process, or compiled Ahead Of Time (AOT compilation) into Java code.
%Either way, while Clojure is being run, it is essentially a Java program, consisting of
%Clojure classes written in Java carrying out their programming.
%\emcomment{hmmm, I am not sure what this (previous) sentence means}
%To the programmer, as much as possible, their code stays within Clojure,
%the Java foundation thoroughly abstracted away.
%\emcomment{Also could be phrased better}
%This allows Clojure to act as a language, and not a library, but it means
% that when an exception is thrown, it's thrown from under Clojure's abstractions
% and so contains a great deal of information about the state of Clojure itself
% during the error, and less information relating to what exactly went wrong in the
% programmers own code.
%\emcomment{Perhaps the key here is that the JVM doesn't perceives the code as Java, and thus just throws a Java exception,
%and Clojure objects are referred to only as Java classes that implement them}
% For a novice programmer, the error message is confusingly laced
%  with Java terms and ideas, and the advanced features of Java that Clojure uses.
%  Furthermore, as the error message is from a Java standpoint, it often misses
%  information useful in debugging, such as actual values involved in some erroring function calls.

\section{Clojure Spec}
A second piece of the information required can be sourced from Clojure Spec,
which is a library within Clojure, dealing with runtime validation of
the structure of data within a program. These specifications can be applied to
 the arguments of function calls, allowing erroneus function calls to have
 error messages that explain how the invocation did not meet the usage pattern
 needed. By providing Spec specifications for the Core library of Clojure,
 we can cover every function that a novice will be able to call, and thus catch
 any miscalled functions with detialed spec error messages.
 This information can be captured by our system, and as a result
 let us access information such as the actual values that caused the errors.
Note that Spec only provides errors for function parameters, syntax errors or
runtime exceptions, we need to source our data from the original Java error messages.
Between Spec error messgages and the original Java error messages, we have enough
information to leverage the tools built by the
 University of Minnesota Morris research group, as soon we can integrate
 our process with the larger Clojure ecosystem.


%Clojure Spec is a library within Clojure allowing runtrime validation
%of the structure of data within the application, with flexibility and precision.
%As it can be used to validate arguments to functions, by providing Spec
% specifications for the Clojure core functions, we can gain Spec errors
% for misapplied fuctions. the data Spec error messages contain lets us build
% error messages with better understanding of the errror that caused them.
%As novice programmers would not use external libraries,
% all functions that they can access would hold Spec specifications.
%\emcomment{If Spec validation is provided for all Clojure core functions,
%this would likely be sufficient for novice programmers since it would be
%rare for them to use functions beyond Clojure core.}
%\emcomment{Note that this only provides errors for function parameters
%type errors; syntax errors cannot be handled via Spec.}
%\emcomment{Also mention here or elsewhere that the research group at UMM has been
%working on providing error messages rewrite system, and the final piece to figure
%out is how to integrate it with Clojure ecosystem.}


\section{Clojure, REPL and the IDE}
Clojure is homoiconic, meaning that a Clojure program is also
vievable as a Clojure datastructure. Evaluating every values in that structure is
equivalent to running the program.
This in part allows a Clojure process to read in data, process it as a program
and display the results back to the programmer and awaiting the next input.
 This common pattern is refered to
as the Read-Eval-Print-Loop (REPL), and programmers use it to test and build up
individual parts of their programs functions. This pattern of development is
referred to as REPL-Driven and is frequently used in Clojure community.
\comment{can I say ''practically all'' here?}
 Most Clojure Intgerated Development Environments (IDEs) feature REPLs for
 user convenience. Outside of developer experimentation, the REPL mechanism is also used
by the Clojure process to load in new Clojure code for execution in a running program,
omitting the Print step as neccesary.
Spanning both evaluation of code, and the preparation of output back to the user,
the REPL is the prime location for our system to be implemented.
However, building our own custom REPL is not advisable, as it limits the
integration possibilities, needing IDEs to be customized to support it.
Instead, the ideal position for our code is within an module loaded
by a popular, expandable REPL implementation.

%As Clojure loads code for exection, it reads in code as a string,
%and then parses it. As Clojure is homoiconic, fully evauluating all values in the datastructure
%is equivalent to running the code.
%As such, a common system in Clojure is the Read-Eval-Print-Loop, whereby a running Clojure
%process reads in data, evauluates it, and then prints the resulting value,
% waiting for the next user input before looping.
%This system lets a user experiment with code as it is written,
% evaluating sub-expressions and building up complex functions.
%Many Clojure IDEs offer a way to interact with a REPL as part of their services,
% and REPL driven development is a common pattern in the Clojure community.
% As the REPL spans both evaluating code and presenting the results of evaluation to users,
% the REPL is a logical place for our transformation of error messages to be implemented.
% To maximize the adaptability of our code, implementing a custom REPL limits integration
% options with IDEs, and with user's workflow in general.
%  Finding an expandable REPL implementation to add our message processing to
%  is ideal.
\section{nREPL}
nREPL is a community standard implementation of the REPL concept,
allowing easy creation, connection and management of REPL servers.
It is integrated with Leiningen, a commonly used project and dependency manager,
 and with many IDES. Moreover, it includes the ability to add custom middleware,
  changing the way the server reads and reacts to code.
 nREPL communicates between its client and server by means of messages.
 By intercepting messages from the server containing error messages,
 we can replace the Java errors with our errors,
  without the user experience changing inadvertently.
This should hold true even for REPLs in integrated into IDEs, as long as they
 use nREPL as a base.

\section{Conclusion}
By harnessing the systems already built for Clojure,
such as Spec and nREPL, we gain needed information.
\emcomment{again, keep in mind that spec only checks function calls. If code is broken
in a way that it never reaches a call (for instance, it has a syntax error), Spec is not helpful.
Our system of replacing error messages handles both cases; you just have to be careful
when you are describing it to not give an impression that it's only Spec that's used.}
By intercepting the display of error messages, we avoid changing users
interactions with the REPL in hidden ways, and allow our code to
be smoothly integrated into other tools as a user added nREPL middleware.
  This alleviates the problems that prevent Clojures
 adoption as a first programming language, and allow it's strengths
 and ties to the Java ecosystem to keep it relevant as the student's skills increase.
\emcomment{This provides the final element for building a system that would allow....}
\emcomment{We should tone it down because Spec-based errors are basically just a proof of concept,
and your demo would also be a proof of concept, rather than a completely fleshed out system - mention the term ``proof of concept'' somewhere.}





\section{Acknowledgments}
This work was supported in part by Morris Academic Partnership (MAP) stipend at UMM.


\bibliographystyle{acm}
\bibliography{mics2018}

\end{document}
