% This is sigproc-sp.tex -FILE FOR V2.6SP OF ACM_PROC_ARTICLE-SP.CLS
% OCTOBER 2002
%
% It is an example file showing how to use the 'acm_proc_article-sp.cls' V2.6SP
% LaTeX2e document class file for Conference Proceedings submissions.
%
%----------------------------------------------------------------------------------------------------------------
% This .tex file (and associated .cls V2.6SP) *DOES NOT* produce:
%       1) The Permission Statement
%       2) The Conference (location) Info information
%       3) The Copyright Line with ACM data
%       4) Page numbering
%
%  However, both the CopyrightYear (default to 2002) and the ACM Copyright Data
% (default to X-XXXXX-XX-X/XX/XX) can still be over-ridden by whatever the author
% inserts into the source .tex file.
% e.g.
% \CopyrightYear{2003} will cause 2003 to appear in the copyright line.
% \crdata{0-12345-67-8/90/12} will cause 0-12345-67-8/90/12 to appear in the copyright line.
%
%
%---------------------------------------------------------------------------------------------------------------
% It is an example which *does* use the .bib file (from which the .bbl file
% is produced).
% REMEMBER HOWEVER: After having produced the .bbl file,
% and prior to final submission,
% you need to 'insert'  your .bbl file into your source .tex file so as to provide
% ONE 'self-contained' source file.
%
% Questions regarding SIGS should be sent to
% Adrienne Griscti ---> griscti@acm.org
%
% Questions/suggestions regarding the guidelines, .tex and .cls files, etc. to
% Gerald Murray ---> murray@acm.org
%
% For tracking purposes - this is V2.6SP - OCTOBER 2002


\documentclass[12pt]{article}

\setlength{\oddsidemargin}{0in}
\setlength{\evensidemargin}{0in}
\setlength{\topmargin}{0in}
\setlength{\headheight}{0in}
\setlength{\headsep}{0in}
\setlength{\textwidth}{6in}
\setlength{\textheight}{9in}
\setlength{\parindent}{0in}

\usepackage{graphicx} %For jpg figure inclusion
\usepackage{times} %For typeface
\usepackage{epsfig}
\usepackage{color} %For Comments
%\usepackage[all]{xy}
\usepackage{float}
%\usepackage{subfigure}
\usepackage{hyperref}
\usepackage{url}
\usepackage{parskip}

%% Elena's favorite green (thanks, Fernando!)
\definecolor{ForestGreen}{RGB}{34,139,34}
\definecolor{BlueViolet}{RGB}{138,43,226}
\definecolor{Coquelicot}{RGB}{255, 56, 0}
\definecolor{Teal}{RGB}{2,132,130}
%Uncomment this if you want to show work-in-progress comments
\newcommand{\comment}[1]{{\bf \tt  {#1}}}
% Uncomment this if you don't want to show comments
%\newcommand{\comment}[1]{}
\newcommand{\emcomment}[1]{\textcolor{ForestGreen}{\comment{Elena: {#1}}}}
\newcommand{\todo}[1]{\textcolor{blue}{\comment{To Do: {#1}}}}
\newcommand{\tscomment}[1]{\textcolor{Teal}{\comment{Tony: {#1}}}}
%%%%%%%%%%%%%%%%%%%%%%%%%%%%%%%%%%%%%%%%%%

\begin{document}
\pagestyle{plain}
%
% --- Author Metadata here ---
%\conferenceinfo{WOODSTOCK}{'97 El Paso, Texas USA}
%\setpagenumber{50}
%\CopyrightYear{2002} % Allows default copyright year (2002) to be
%over-ridden - IF NEED BE.
%\crdata{0-12345-67-8/90/01}  % Allows default copyright data
%(X-XXXXX-XX-X/XX/XX) to be over-ridden.
% --- End of Author Metadata ---

\title{TBA}
%\subtitle{[Extended Abstract \comment{DO WE NEED THIS?}]
%\titlenote{}}
%
% You need the command \numberofauthors to handle the "boxing"
% and alignment of the authors under the title, and to add
% a section for authors number 4 through n.
%
% Up to the first three authors are aligned under the title;
% use the \alignauthor commands below to handle those names
% and affiliations. Add names, affiliations, addresses for
% additional authors as the argument to \additionalauthors;
% these will be set for you without further effort on your
% part as the last section in the body of your article BEFORE
% References or any Appendices.

\author{
Charlot Shaw \\
Computer Science Discipline \\
University of Minnesota Morris\\
Morris, MN 56267\\
shawx538@morris.umn.edu
}
\maketitle
\thispagestyle{empty}
%\alcomment{Should these say @morris.umn.edu?}

\section*{\centering Abstract}


\newpage
\setcounter{page}{1}

\section{Introduction}
Clojure, amongst other functional languages have gained attention in recent years
in part for their management of state and concurrency.
Clojure, which runs on the Java Virtual Machine (JVM) has a number of advantages for beginning students.
Its Lisp style syntax and elegant core work \emcomment{not sure what ``core work'' means}
 let beginners learn easily, while
its ties to the JVM keep it relevant as students progress into later courses.
From the University of Minnesota Morriss's perspective 
\emcomment{This is not just UMM perspective, this is a general barrier in using it in a first programming class}, it has a significant flaw in the form of error messages.
As Clojure code runs in the JVM, its errors take the structure and terminology of Java error messages,
and so are confusing to new students. They can understand the source of the error,
but not how the system presents it to them.
For example, a user who accidentally called addition on an boolean needs
 to understand classes, casting, and the classes \emcomment{saying ``classes'' twice} involved to understand "java.lang.classCastException".
\emcomment{The previous sentence is comnfusing; rephrase}
 In contrast, a message that says "true cannot be added to 7" does not require additional knowledge from the user.
 In order to overcome these problems,
  we demonstrate an integration with commonly used tools in the Clojure community
 to transform error messages into simplified forms before they are presented to the user.

\cite{Hickey:2008}

\section{Error Messages in Clojure}


\section{Clojure, REPL and the IDE}
Clojure is hosted and interpretted in the JVM, with Clojure code either being
loaded by a Clojure process, or compiled Ahead Of Time \emcomment{not sure if it's a term; perhaps say ``ahead of loading''} into Java code.
As Clojure loads code for exection, it reads in code as a string,
and then parses it. As Clojure is homoiconic, fully evauluating all values in the datastructure
is equivalent to running the code.
As such, a common system in Clojure is the Read-Eval-Print-Loop, whereby a running Clojure
process reads in data, evauluates it, and then prints the resulting value, waiting for the next user input before looping.


\section{nREPL}

\section{Acknowledgments}
This work was supported in part by Morris Academic Partnership (MAP) stipend at UMM.


\bibliographystyle{acm}
\bibliography{mics2018}

\end{document}
